\documentclass[modern]{aastex631}

% Affiliations
\newcommand{\cca}{Center for Computational Astrophysics, Flatiron Institute, 162 Fifth Avenue, New York, NY 10010, USA}
\newcommand{\sbu}{Department of Physics and Astronomy, Stony Brook University, Stony Brook, NY 11794, USA}
\newcommand{\pa}{Department of Physics and Astronomy, Northwestern University, 2145 Sheridan RD, Evanston, IL 60208, USA}
\newcommand{\ciera}{Center for Interdisciplinary Exploration and Research in Astrophysics (CIERA), Northwestern University, 1800 Sherman Ave, Evanston, IL 60201, USA}

\begin{document}
\title{Hiding Out at the Low End}

\author[0000-0003-1540-8562]{Will M. Farr}
\email{will.farr@stonybrook.edu}
\email{wfarr@flatironinstitute.org}
\affiliation{\sbu}
\affiliation{\cca}

\author[0000-0001-9236-5469]{Vassiliki Kalogera}
\email{vicky@northwestern.edu}
\affiliation{\pa}
\affiliation{\ciera}

\begin{abstract}
    It is not known whether there is a continum of masses of compact objects
    formed from stellar collapse from the heaviest neutron stars to the lightest
    black holes or whether there is a gap in the mass spectrum between these
    classes of objects.  The presence or absence of a mass gap has implications
    for the supernova mechanism, as well as being a fundamental property of the
    compact object mass function.  In X-ray binaries containing black holes a
    gap is observed, but it is not known whether this is representative of a
    true gap in the mass function or due to selection effects or systematic
    biases in mass estimation.  A small number of black holes have been observed
    with luminous companions in non-interacting orbits, but as yet the sample is
    too small to assess the existence of a gap, and in any case selection
    effects in this sample are hard to quantify.  Binary black hole mergers
    detected from gravitational waves in the GWTC-3 transient catalog furnish a
    large sample of several tens of low-mass black holes with a well-understood
    selection function.  Here we analyze the 15 GWTC-3 merger events with black
    hole masses in $3 \, M_\odot < m_2 < m_1 < 20 \, M_\odot$ in detail to
    uncover the structure of the low-mass black hole mass function in these
    systems.  Using several flexible parameterized models for the mass function,
    we find a sharp peak (width $\lesssim 20\%$) in the mass function at $m =
    9.47^{+0.52}_{-0.58} \, M_\odot$, associated with merger rates $m_1 m_2
    \mathrm{d} N / \mathrm{d} m_1 \mathrm{d} m_2 \mathrm{d} V \mathrm{d} t =
    270^{+270}_{-150} \, \mathrm{Gpc}^{-3} \, \mathrm{yr}^{-1}$.  The mass
    function falls by at least an order of magnitude both below and above this
    peak.  Toward the lowest masses, the mass function may or may not flatten;
    we find that the $1\%$ black hole mass in our most flexible model is
    $m_{1\%} = 3.28^{+1.19}_{-0.13} \, M_\odot$. In other words, this sample of
    low-mass black holes does not require a mass gap but may permit one;
    observations in the currently-ongoing ``O4'' observation run should
    distinguish these possibilities.  Toward higher masses, the mass function
    declines to our upper limit $m = 20 \, M_\odot$ steeply, with power law
    slopes $\mathrm{d} N / \mathrm{d} m \sim m^{-\alpha}$, $\alpha =
    -4.9^{+3.1}_{-3.3}$.  We find no evidence for or against a secondary,
    sharply-peaked population of black holes around $m \simeq 9 \, M_\odot$ as
    has been suggested by several models of stellar evolution.
\end{abstract}

\section{Introduction}

\end{document}
